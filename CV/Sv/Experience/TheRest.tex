\begin{minipage}[t]{0.9\textwidth}
	\subsection*{\textcolor{colorTitelErfarenhet}{CGI, Server hanterare klient – 2023/08 till 2023/09}}
	Björn var ensam utvecklare för detta projekt för att hantera en servers olika funktioner som 
	varierar från övervakning av tjänster till ett gränssnitt för filhämtning. Användargränssnittet 
	använder sig av skalbar vektorgrafik för att uppnå en bättre användare vänlighet och estisk. 
	Utvecklingen implementerades enligt följa alla stegen i en CI/CD process. 
	
	\vspace{5pt}\textit{Tekniker som använts: Node.js, TypeScript, JavaScript, Vue3, HTML, CSS, GIT, Bash, Linux och SVG.}
	
	\vspace{15pt} % Justera vertikal placering av minipage
	
    \subsection*{\textcolor{colorTitelErfarenhet}{CGI, Blekingetrafiken buss APP – 2023/08 till 2023/08}}
    Björn var ensam utvecklare där han gjorde alla steg i processen från CI/CD, implementation, 
    Testning, publicering av projektet. Projektet hjälper Björn och andra att kolla upp transport 
    till och från CGI kontoret för mer effektiv tidshantering. Appen använder sig av blekingetrafikens 
    API för att få data om dom olika resmålen.
    
    \vspace{5pt}\textit{Tekniker som använts: Python3, GIT, Bash, Linux, Dockor och Jenkins.}
    
    \vspace{15pt} % Justera vertikal placering av minipage
    
    \subsection*{\textcolor{colorTitelErfarenhet}{CGI, Flexmeister – 2023/06 till 2023/08}}
    För att på ett mer effektivare set logga och hantera anställdas flex tider var Björn med och 
    skapade en flex-tids hanterar app som har en webbaserad klient och en server med ett Rest API som är 
    byggt för att Kunnas byggas ut om fler sorters klienter skulle behövas. Utvecklingen implementerades 
    enligt följa alla stegen i en CI/CD process.
  
    \vspace{5pt}\textit{Tekniker som använts: Python3, GIT, Bash, Linux, Dockor och Jenkins.}
  
    \vspace{15pt} % Justera vertikal placering av minipage
    \subsection*{\textcolor{colorTitelErfarenhet}{CGI, Studentprojekt i samarbete med BTH, mjukvaruutvecklare - 2020/02 till 2020/06}}
    Tillsammans med sitt team utvecklade Björn en ritningshantering för marinen. 
    I projektet användes agila arbetsflöden med användning av verktyget Jira. 
    Utvekling inom Java, Spring Boot, Unit Testing, Integration tests.
    
    \vspace{5pt}\textit{Tekniker som använts: Java, JavaScript, Spring Boot, Visual Code, Gradel, Git, Jira, Unit Testing och Integration tests.} 
    
    \vspace{15pt} % Justera vertikal placering av minipage
    \subsection*{\textcolor{colorTitelErfarenhet}{SIGMA, Webb/mjukvaruutveckling - 2019/02 till 2019/07}}
    Under denna tid så utvecklade Björn tillsammans med sitt team ett transportsystem 
    till Karlskrona kommun, som nu används av kommunanställda för att boka transporter. 
    Han var en fullstack utvecklare med fokus på testning och front-end utveckling.
    
    \vspace{5pt}\textit{Tekniker som använts: JavaScript, Dart, NodeJs, Gradle, MariaDB, Azure, Google Cloud, Git, React.js och Maven, Docker.} 
    
  
    \vspace{15pt} % Justera vertikal placering av minipage
    \subsection*{\textcolor{colorTitelErfarenhet}{Sjobeck Prime, Serveransvarig/mjukvaruutveckling - 2014/01 till 2014/08}}
    Under sin tid på Sjobeck Prime satte Björn upp Linux-baserade utvecklingsservrar samt utvecklade 
    prototyper för Iphone. Men han var även ansvarig för de interna servrarna och tjänsterna.
    
    \vspace{5pt}\textit{Tekniker som använts: Java, Bash, Objekt-C, MySQL, XCode, Linux, Git och Jira} 
      
  
    \vspace{15pt} % Justera vertikal placering av minipage
    \subsection*{\textcolor{colorTitelErfarenhet}{Simrishamn kommun, IT avdelning - 2011/03 till 2011/09}}
    Björn utvecklade kommunens rapporteringssystem för hemtjänsten, mestadels i programmeringsspråket .NET.
    
    \vspace{5pt}\textit{Tekniker som använts: MySQL, .NET, JavaScript, Visual Studio och Git.}
      
  
    \vspace{15pt} % Justera vertikal placering av minipage
    \subsection*{\textcolor{colorTitelErfarenhet}{Crunch Fish, APU/ Android apputveckling - 2009/01 till 2010/01}}
    Björn arbetade med att utveckla en musikinriktad app för Android. Han var även med och gjorde prototyper för framtida projekt.
    
    \vspace{5pt}\textit{Tekniker som använts: Java, XML, HTML och Android Studio.}
\end{minipage}%