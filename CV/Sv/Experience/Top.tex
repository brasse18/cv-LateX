\subsection*{\textcolor{colorTitelErfarenhet}{Nortic, Fortnox inporter – 2023/09 till 2024/03}}
Björn var ansvarig för att integrera Nortics interna system för att effektivt exportera transaktioner till Fortnox-plattformen. 
Under denna tid implementerades olika AWS-tjänster, inklusive Lambda-funktioner, för att optimera och parallellisera arbetsflödet med hjälp av MQTT-kösystem. 
Python utnyttjades omfattande i projektet för att säkerställa sömlös funktionalitet.

Vidare arbetade Björn aktivt med Google och Apple Wallets för att möjliggöra export av biljetter från Nortics interna system. 
Detta gjorde det smidigt för användarna att visa sina biljetter direkt i sina mobila enheter. 
I denna process användes också en rad olika AWS-tjänster samt teknologier som Node.js och TypeScript för att upprätthålla en hög standard och säkerställa optimal prestanda.

\vspace{5pt}\textit{Tekniker som använts: Java, Node.js, TypeScript, JavaScript, Python, HTML, CSS, GIT, Bash, Linux och serverless (AWS)}

\vspace{15pt} % Justera vertikal placering av minipage
\subsection*{\textcolor{colorTitelErfarenhet}{CGI, Server hanterare klient – 2023/08 till 2023/09}}
Björn var ensam utvecklare för detta projekt för att hantera en servers olika funktioner som 
varierar från övervakning av tjänster till ett gränssnitt för filhämtning. Användargränssnittet 
använder sig av skalbar vektorgrafik för att uppnå en bättre användare vänlighet och estisk. 
Utvecklingen implementerades enligt följa alla stegen i en CI/CD process. 

\vspace{5pt}\textit{Tekniker som använts: Node.js, TypeScript, JavaScript, Vue3, HTML, CSS, GIT, Bash, Linux och SVG.}
