Björn är en passionerad utvecklare med drygt 3 års erfarenhet av fullstack mjukvaruutveckling, både i större team och som enskild utvecklare. Han har arbetat och studerat inom olika system relaterade till webbutveckling och molntjänster. Tidigare har han implementerat flera end-to-end-lösningar med API
och databasanslutningar, inklusive vidareutveckling av Simrishamns kommuns .NET-baserade hemtjänstsystem. Under sin tid på SIGMA arbetade han också med Azure och dess tjänster för att utveckla ett transportsystem.

Under sin anställning på Nortic fokuserade Björn på att integrera egna system för transaktionsexport till Fortnox, genom att utnyttja olika AWS-tjänster, inklusive Lambda-funktioner och MQTT-kösystem. Python spelade en central roll i dessa projekt. Dessutom bidrog han till integrationen av Google och Apple Wallets för biljettexporter, med användning av AWS-tjänster tillsammans med Node.js och TypeScript.

Björn har starka problemlösningsförmågor och strävar kontinuerligt efter att lära sig nya tekniker och verktyg. Hans bakgrund inom mjukvaruutveckling inkluderar även andra teknologier och verktyg såsom JavaScript, Vue, HTML, CSS, Bootstrap, React.js, MongoDB, Google Cloud och Git, vilket gör honom anpassningsbar och kapabel att arbeta med de flesta teknologier.