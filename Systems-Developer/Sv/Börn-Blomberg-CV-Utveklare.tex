\documentclass{article}
\usepackage{booktabs}
\usepackage{geometry}
\usepackage{xcolor}
\usepackage{colortbl}
\usepackage{graphicx}
\usepackage{wrapfig}
\usepackage{caption}
\usepackage{titlesec}
\usepackage{enumitem}
\usepackage[export]{adjustbox} % Lägg till adjustbox-paketet

\newgeometry{left=1.5cm, right=1.5cm, top=1.5cm}
\setlist[itemize]{itemsep=-2pt}
\definecolor{colorBlue}{HTML}{646EDB}
\definecolor{colorBlueTwo}{HTML}{64AADB}
\definecolor{colorRed}{HTML}{DB646E}
\definecolor{colorGreen}{HTML}{6FDB64}
\definecolor{colorBron}{HTML}{DB9564}

\definecolor{colorMainText}{HTML}{222222}
\definecolor{colorSubText}{HTML}{666666}

\definecolor{colorTitelErfarenhet}{HTML}{DB9564}

% Definiera färgen för subrubrikerna

%\setlength{\arrayrulewidth}{1pt} % Ta bort tjockleken på tabellens linjer

\begin{document}

\begin{minipage}[t]{0.7\textwidth}
\titleformat{\section}{\Huge\bfseries}{\thesection}{1em}{}
\titlespacing{\subsection}{0pt}{*0.5}{*0.5}
\section*{\textcolor{colorBlue}{Björn Blomberg}}
\subsection*{\textcolor{colorTitelErfarenhet}{Systemutvecklare}}
\subsection*{3 års erfarenhet av systemutveckling}
\rule{10cm}{0.4pt}
\titleformat{\section}{\Large\bfseries}{\thesection}{1em}{}
\vspace{1.5cm}
\end{minipage}%
\begin{minipage}[t]{0.3\textwidth}
\vspace{-10pt} % Justera vertikal placering av minipage
\end{minipage}%

\noindent
\begin{minipage}[t]{0.7\textwidth}
\vspace{-20pt} % Justera vertikal placering av minipage
\section*{\textcolor{colorBlue}{Profil}}
Björn är en passionerad utvecklare med strax över 3 års erfarenhet av fullstack-
mjukvaruutveckling i både större teams och som enskild utvecklare. 
Han har jobbat och studerat i många olika system som rör webbutveckling och molntjänster. 
Sedan tidigare har han gjort flertalet end-to-end lösningar med API:er och databaskopplingar, 
bland annat vidareutveckling av Simrishamns kommuns .NET-baserade hemtjänstsystem. 
Under sin tid på SIGMA har han även suttit med Azure och dess tjänster för att ta fram ett transportsystem.

\vspace{10pt}
Björn har en stark problemlösningsförmåga, och strävar kontinuerligt efter att lära 
sig nya tekniker och verktyg. Hans bakgrund inom mjukvaruutveckling inkluderar 
också andra tekniker och verktyg som JavaScript, Vue, HTML, CSS, Bootstrap, React.js, MongoDB,
 Elastic Search, Googel-cloud och Git, vilket gör honom anpassningsbar och 
 kapabel att jobba med de flesta teknologier.

\vspace{10pt} % Justera vertikal placering av minipage
\section*{\textcolor{colorBlue}{Erfarenhet}}

\subsection*{\textcolor{colorTitelErfarenhet}{Nortic, Fortnox inporter – 2023/09 till 2024/03}}

Björn var ansvarig för att integrera Nortics interna system för att effektivt exportera transaktioner till Fortnox-plattformen. Under denna tid implementerades olika AWS-tjänster, inklusive Lambda-funktioner, för att optimera och parallellisera arbetsflödet med hjälp av MQTT-kösystem. Python utnyttjades omfattande i projektet för att säkerställa sömlös funktionalitet.

Vidare arbetade Björn aktivt med Google och Apple Wallets för att möjliggöra export av biljetter från Nortics interna system. Detta gjorde det smidigt för användarna att visa sina biljetter direkt i sina mobila enheter. I denna process användes också en rad olika AWS-tjänster samt teknologier som Node.js och TypeScript för att upprätthålla en hög standard och säkerställa optimal prestanda.

\vspace{5pt}\textit{Tekniker som använts: Java, Node.js, TypeScript, JavaScript, Python, HTML, CSS, GIT, Bash, Linux och serverless (AWS)}

\vspace{15pt} % Justera vertikal placering av minipage
\subsection*{\textcolor{colorTitelErfarenhet}{CGI, Server hanterare klient – 2023/08 till 2023/09}}
Björn var ensam utvecklare för detta projekt för att hantera en servers olika funktioner som 
varierar från övervakning av tjänster till ett gränssnitt för filhämtning. Användargränssnittet 
använder sig av skalbar vektorgrafik för att uppnå en bättre användare vänlighet och estisk. 
Utvecklingen implementerades enligt följa alla stegen i en CI/CD process. 

\vspace{5pt}\textit{Tekniker som använts: Node.js, TypeScript, JavaScript, Vue3, HTML, CSS, GIT, Bash, Linux och SVG.}


\end{minipage}%
\hfill
\begin{minipage}[t]{0.28\textwidth}
\begin{minipage}[t]{0.8\textwidth}
\vspace{-140pt} % Justera vertikal placering av minipage
  \includegraphics[width=\linewidth]{../../me.png}
  \label{fig:bild}
\end{minipage}
\vspace{-10pt} % Justera vertikal placering av minipage
\subsection*{\textcolor{colorBlue}{Kvalitiationer}}
\begin{itemize}
  \item Nätverk
  \item Problemlösning
  \item Mjukvarudesign
  \item Agila arbetsflöden
  \item Serverunderhåll
  \item Linux system / Bash
  \item Assembler AT\&T/ GNU
  \item C / C++ / C\# / Objekt-C
  \item Java / JavaScript / .NET
  \item Python / GO
  \item HTML / CSS / PHP /SQL
  \item Spring Boot / Gradel
  \item Unit Testing
  \item Integration Teststing
  \item Windows Subsystem Linux (WSL)
\end{itemize}
\vspace{-10pt} % Justera vertikal placering av minipage
\subsection*{\textcolor{colorBlue}{Verktyg \& programvara}}
\begin{itemize}
  \item Jira
  \item Git
  \item Visual Studeo
  \item Anddroid Studeo
  \item IntelliJ
  \item Eclipse
\end{itemize}
\vspace{-10pt} % Justera vertikal placering av minipage
\subsection*{\textcolor{colorBlue}{Språk}}
Svenska, Engelska
\end{minipage}

\begin{minipage}[t]{0.7\textwidth}
	
	
  \vspace{15pt} % Justera vertikal placering av minipage
  \subsection*{\textcolor{colorTitelErfarenhet}{CGI, Blekingetrafiken buss APP – 2023/08 till 2023/08}}
  Björn var ensam utvecklare där han gjorde alla steg i processen från CI/CD, implementation, 
  Testning, publicering av projektet. Projektet hjälper Björn och andra att kolla upp transport 
  till och från CGI kontoret för mer effektiv tidshantering. Appen använder sig av blekingetrafikens 
  API för att få data om dom olika resmålen.
  
  \vspace{5pt}\textit{Tekniker som använts: Python3, GIT, Bash, Linux, Dockor och Jenkins.}
  
  \vspace{15pt} % Justera vertikal placering av minipage
  
  \subsection*{\textcolor{colorTitelErfarenhet}{CGI, Flexmeister – 2023/06 till 2023/08}}
  För att på ett mer effektivare set logga och hantera anställdas flex tider var Björn med och 
  skapade en flex-tids hanterar app som har en webbaserad klient och en server med ett Rest API som är 
  byggt för att Kunnas byggas ut om fler sorters klienter skulle behövas. Utvecklingen implementerades 
  enligt följa alla stegen i en CI/CD process.

  \vspace{5pt}\textit{Tekniker som använts: Python3, GIT, Bash, Linux, Dockor och Jenkins.}

  \vspace{15pt} % Justera vertikal placering av minipage
  \subsection*{\textcolor{colorTitelErfarenhet}{CGI, Studentprojekt i samarbete med BTH, mjukvaruutvecklare - 2020/02 till 2020/06}}
  Tillsammans med sitt team utvecklade Björn en ritningshantering för marinen. 
  I projektet användes agila arbetsflöden med användning av verktyget Jira. 
  Utvekling inom Java, Spring Boot, Unit Testing, Integration tests.
  
  \vspace{5pt}\textit{Tekniker som använts: Java, JavaScript, Spring Boot, Visual Code, Gradel, Git, Jira, Unit Testing och Integration tests.} 
  
  \vspace{15pt} % Justera vertikal placering av minipage
  \subsection*{\textcolor{colorTitelErfarenhet}{SIGMA, Webb/mjukvaruutveckling - 2019/02 till 2019/07}}
  Under denna tid så utvecklade Björn tillsammans med sitt team ett transportsystem 
  till Karlskrona kommun, som nu används av kommunanställda för att boka transporter. 
  Han var en fullstack utvecklare med fokus på testning och front-end utveckling.
  
  \vspace{5pt}\textit{Tekniker som använts: JavaScript, Dart, NodeJs, Gradle, MariaDB, Azure, Google Cloud, Git, React.js och Maven, Docker.} 
  

  \vspace{15pt} % Justera vertikal placering av minipage
  \subsection*{\textcolor{colorTitelErfarenhet}{Sjobeck Prime, Serveransvarig/mjukvaruutveckling - 2014/01 till 2014/08}}
  Under sin tid på Sjobeck Prime satte Björn upp Linux-baserade utvecklingsservrar samt utvecklade 
  prototyper för Iphone. Men han var även ansvarig för de interna servrarna och tjänsterna.
  
  \vspace{5pt}\textit{Tekniker som använts: Java, Bash, Objekt-C, MySQL, XCode, Linux, Git och Jira} 
    

  \vspace{15pt} % Justera vertikal placering av minipage
  \subsection*{\textcolor{colorTitelErfarenhet}{Simrishamn kommun, IT avdelning - 2011/03 till 2011/09}}
  Björn utvecklade kommunens rapporteringssystem för hemtjänsten, mestadels i programmeringsspråket .NET.
  
  \vspace{5pt}\textit{Tekniker som använts: MySQL, .NET, JavaScript, Visual Studio och Git.}
    

  \vspace{15pt} % Justera vertikal placering av minipage
  \subsection*{\textcolor{colorTitelErfarenhet}{Crunch Fish, APU/ Android apputveckling - 2009/01 till 2010/01}}
  Björn arbetade med att utveckla en musikinriktad app för Android. Han var även med och gjorde prototyper för framtida projekt.
  
  \vspace{5pt}\textit{Tekniker som använts: Java, XML, HTML och Android Studio.}

\end{minipage}%
\vspace{40pt} % Justera vertikal placering av minipage

\begin{minipage}[t]{0.4\textwidth}

\section*{\textcolor{colorBlue}{Tekniska specialiseringar}}

\begin{description}
  \item [ Frontend utveckling ]
  \item [ Backend utveckling ]
  \vspace{-5pt}\item [ Linux \& DevOps ] 
  \vspace{-5pt}\item [ CI/CD ]
  \vspace{-5pt}
\end{description}

\end{minipage}%
\hfill
\begin{minipage}[t]{0.6\textwidth}

\section*{\textcolor{colorBlue}{Utbildning}}

\begin{description}
  \item [ Software engineering, Kandidatexamen, Blekinge tekniska högskola ] 2017 - pågående
  \item [ Tekniskt Basår, Kandidatexamen, Blekinge tekniska högskola ] 2015 - 2016
  \item [ Javautveckling (1 år), Yrkesutbildning, Malmö ] 2014 - 2015

  \item [ Linux systemspecialist (2 år), Yrkesutbildning, Helsingborg ] 2012 - 2014 
  
  Linux system och dess uppbyggnad, virtualisering och klustring av 
 system och tjänster, användarhantering i en Linux miljö, web server 
 service-hantering och underhåll.
\end{description}

\end{minipage}%
\vspace{10pt}

\begin{tabular}{|l|l|l|}
\hline
\rowcolor{colorBlue}
\multicolumn{1}{|l|}{\textcolor{white}{\textbf{Kompetens}}} & \multicolumn{1}{l|}{\textcolor{white}{\textbf{Antal år}}} & \multicolumn{1}{l|}{\textcolor{white}{\textbf{Skicklighetsgrad}}} \\
\hline
\rowcolor{colorBlueTwo}
\multicolumn{3}{|l|}{\textcolor{white}{\textbf{Teknisk kunskap}}} \\
\hline
C++ & 6 & 3 \\
\hline
C & 2 & 2 \\
\hline
C\# & 2 & 2 \\
\hline
Objekt-C & 1 & 1 \\
\hline
Linux & 8 & 4 \\
\hline
Bash & 8 & 4 \\
\hline
Unit Testing & 3 & 3 \\
\hline
Integration Testing & 2 & 2 \\
\hline
Load Testing & 2 & 2 \\
\hline
Assembler AT\&T & 3 & 3 \\
\hline
Azure & 1 & 1 \\
\hline
AWS & 1 & 1 \\
\hline
Spring Boot & 1 & 2 \\
\hline
Java & 2 & 3 \\
\hline
JavaScript & 3 & 3 \\
\hline
TypeScript & 1 & 2 \\
\hline
.NET & 1 & 1 \\
\hline
Python & 1 & 2 \\
\hline
GO & 1 & 1 \\
\hline
HTML & 8 & 3 \\
\hline
CSS & 8 & 2 \\
\hline
PHP & 1 & 1 \\
\hline
SQL & 3 & 2 \\
\hline
Docker & 1 & 2 \\
\hline
Chat-GPT & 1 & 2 \\
\hline
MongoDB & 1 & 2 \\
\hline
Postgresql&1 & 2 \\
\hline
Boostrap & 1 & 2 \\
\hline
Jenkins & 1 & 1 \\
\hline
Marvin & 1 & 1 \\
\hline
Gradel & 1 & 1 \\
\hline
Hibernate & 1& 1 \\
\hline
Windows Subsystem Linux (WSL1, WSL2) & 3 & 3 \\
\hline
\rowcolor{colorBlueTwo}
\multicolumn{3}{|l|}{\textcolor{white}{\textbf{Applikationskunskap}}} \\
\hline
Eclipse & 1 & 2 \\
\hline
Visual Studio & 10 & 3 \\
\hline
Android studio & 1 & 2 \\
\hline
IntelliJ & 3 & 3 \\
\hline
Git & 5 & 3 \\
\hline
\rowcolor{colorBlueTwo}
\multicolumn{3}{|l|}{\textcolor{white}{\textbf{IT disciplines}}} \\
\hline
Systemutvecklare & 3 & 3 \\
\hline
Mjukvaruutvecklare & 6 & 3 \\
\hline
Embedded Software Development & 1 & 3 \\
\hline
CI/CD processer & 3 & 3 \\
\hline
Agile ITIL arbetsflöden & 3 & 3 \\
\hline
\rowcolor{colorBlue}
\multicolumn{3}{|l|}{\textcolor{white}{\textbf{Tal Språk}}} \\
\hline
Svenska & 30 & 4 \\
\hline
Engelska & 25 & 4 \\
\hline
\multicolumn{3}{l}{\textbf{\*Skicklighetsgrad: 1 = Basic, 2 = Intermediate, 3 = Advanced, 4 = Expert}} \\


\end{tabular}


\end{document}